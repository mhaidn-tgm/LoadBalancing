\documentclass[12p]{article}

\usepackage[english]{babel}
\usepackage[utf8x]{inputenc}
\usepackage[colorinlistoftodos]{todonotes}
\usepackage{fancyhdr} %Package to configure headings and footer
\usepackage{lastpage} %Needed to display last page (total amount of pages)
\usepackage{listings}

% pagelayout
\usepackage[
top    = 2.75cm,
bottom = 2.00cm,
left   = 2.50cm,
right  = 2.00cm]{geometry}
\setcounter{secnumdepth}{4}

% header
\pagestyle{fancy}
\fancyhead[L]{\today}
\fancyhead[R]{Load Balancing - Gliederung}

%footer
\fancyfoot[L]{Haidn, Schrack}
\fancyfoot[C]{5A HIT}
\fancyfoot[R]{Seite \thepage/\pageref{LastPage}}

%title page
\author{Martin Haidn, Nikolaus Schrack}
\title{Load Balancing - Structure\\SYT - 5A HIT}
\date{\today}

%Glossary
\usepackage{glossaries}
\makeglossaries
\newglossaryentry{mac}{name=MAC, description={Media Access Controll}}

\begin{document}
	\maketitle
	
	\newpage
	\tableofcontents
	
	\newpage
	\section{Instruction}
	\subsection{The Need and Goals for Load Balancing}
	This section should describe what's the aim of using load distribution and why or respectively where it's needed.
	\subsection{Use Cases and Examples}
	This section should pick up the significant points from the "Needs and Goals" and bring them in a relation with specific, real examples.
	\subsection{Applications}
	An overview about the common used applications for load distribution.
	
	\newpage
	\section{Basic Concepts}
	\subsection{Networking Fundamentals}
	The OSI model contains seven layers and every single one provides it's own functionality and data.\\
	If we take a closer look on the deeper layers like data link and network, which is representative for layer two and three, we can see that their header information contains IP and MAC addresses. These addresses can be used to decide where a package has to be send when it's revived by a switch.\\
	This basic concept of routing packages builds the fundament for load balancing. It's about making a decision if, and where the data has to go. \cite{lb_SFC}
	\subsection{Higher Layered Distribution}
	Description how load distribution works on OSI-Layers six and seven.
	\subsection{Load-Distribution Methods}
	Summary of common load distribution Methods, their benefits and disadvantages.
	
	\newpage
	\section{Advanced Concepts}
	\subsection{Session Persistence}
	Reasons and benefits of using Session Persistence to track and store session data.
	\subsection{URL Switching}
	The flexibility of layer seven load balancing and the included url switching.
	\subsection{Network-Address Translation}
	Fast Layer 4 load balancing and the appliance as default gateway.
	
	\newpage
	\section{Scheduling Algorithms}
	\subsection{Weighted Balance}
	Ways to guarantee a weighted balance in busy systems.
	\subsection{Priority}
	The meaning of priorities concerning the process of load balancing and how to route traffic to a preferred link, as long it's available.
	\subsection{Overflow}
	How to prevent traffic flow from slowing down when the connection runs out of available bandwidth.
	\subsection{Persistance}
	Eliminate session termination issue for HTTPS, E-banking, and other secure websites.
	\subsection{Round-Robin}
	A closer explanation to the scheduling procedure "Round Robin"
	
	\newpage
	\section{Caches}
	\subsection{Definition}
	Define what a cache is for when we talk about load balancing.
	\subsection{Types}
	The different types of caches and their usage as well as benefits and disadvantages.
	\subsection{Deployment}
	Examples and explanation how to deploy load distribution using caches.
	
	\newpage
	\section{Problems}
	\subsection{Mega Proxy Session}
	Problems triggered through the use of Mega Proxys on the client site.
	
	\newpage
	\printglossaries
	
	\newpage
	\bibliographystyle{plain}
	\bibliography{Loadbalancing_Haidn_Schrack.bib}
	
	\newpage
	\section{Sources}
	\todo{Please note that this is just an overview about our collected references so far, to find them in the TU-Library. We'll add our last visits during this elaberation.}
	Titel:    Load balancing servers, firewalls, and caches : [timely, practical, reliable]\\
	Autor:    Chandra Kopparapu\\
	Jahr:    2002\\
	ISBN:    ISBN 0-471-41550-2\\
	TUWS:    DAT:964, DAT:224\\
	\\
	Titel:    Dynamic load balancing : an overview\\
	Autor:    Arnold R. Krommer ; Christoph W. Ueberhuber\\
	Jahr:    1992\\
	ISBN:     -\\
	TUWS:    DAT:351\\
	\\
	Titel:    Dynamischer Lastausgleich in Parallelrechnersystemen : genetische Algorithmen und eine spezielle Rechnerstruktur\\
	Autor:    Michael Witt\\
	Jahr:    1997\\
	ISBN:    -\\
	TUWS:    -\\
	\\
	Titel:    Optimal load balancing in distributed computer systems\\
	Autor:    Hisao Kameda\\
	Jahr:    1997\\
	ISBN:    ISBN 3-540-76130-6\\
	TUWS:     -\\
	\\
	\\
	Titel:     Server Load Balancing\\
	Autor:     Tony Bourke\\
	Jahr:     August 2001\\
	0-596-00050-2, Order Number: 0502\\
	200 pages, 34.95 USD\\
	Link: http://oreilly.com/catalog/serverload/chapter/ch07.html\\
	\\
	\\
	Onlinequellen:\\
	\\
	Name:    Optimal Load Balancing in Distributed Computer Systems\\
	Link:    http://bookzz.org/book/2092081/f777c1\\
	\\
	Name:    Spectral Methods for Efficient Load Balancing Strategies\\
	Link:    http://cs.emis.de/LNI/Dissertation/Dissertation3/GI-Dissertations.03-3.pdf\\
	\\
	Name:    Lastverteilung auf dem Konzept des virtuellen Servers\\
	Link:    http://www.nm.ifi.lmu.de/pub/Fopras/fikr02/PDF-Version/fikr02.pdf\\
	\\
	Name: Dynamic Load Balancing and Scheduling\\
	Link: http://www2.cs.uni-paderborn.de/cs/ag-monien/RESEARCH/LOADBAL/\\
	\\
	
\end{document}